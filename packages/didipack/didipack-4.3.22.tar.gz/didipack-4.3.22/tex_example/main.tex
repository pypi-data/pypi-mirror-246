% $Id: jfsample.tex,v 19:a118fd22993e 2013/05/24 04:57:55 stanton $
\documentclass[11pt]{article}

% DEFAULT PACKAGE SETUP
%\usepackage[printfigures]{figcaps}
\usepackage{setspace,graphicx,epstopdf,amsmath,amsfonts,amssymb,amsthm,versionPO}
\usepackage{marginnote,datetime,enumitem,subfigure,rotating,fancyvrb}
%\usepackage{marginnote,datetime,enumitem,subfigure,fancyvrb}
\usepackage{hyperref,float}
\usepackage{booktabs}
\usepackage[longnamesfirst]{natbib}
\usepackage{mathtools}
\usepackage{lscape}
\usepackage[toc,page]{appendix}
% custom antoine stuff
\def\rot{\rotatebox}
\usepackage{adjustbox}

\usepackage[utf8]{inputenc}
\usepackage{newunicodechar}
\usepackage{tikz}
\usetikzlibrary{positioning}
\usepackage{forest}
\newtheorem{prop}{Proposition}
\usetikzlibrary{trees}

\usdate

% These next lines allow including or excluding different versions of text
% using versionPO.sty

\excludeversion{notes}          % Include notes?
\includeversion{links}          % Turn hyperlinks on?

% Turn off hyperlinking if links is excluded
\iflinks{}{\hypersetup{draft=true}}

% Notes options
\ifnotes{%
\usepackage[margin=1.0in,paperwidth=10in,right=2.5in]{geometry}%
\usepackage[textwidth=1.4in,shadow,colorinlistoftodos]{todonotes}%
}{%
\usepackage[margin=1in]{geometry}%
\usepackage[disable]{todonotes}%
}

% make footnote pripritary so we don't split them from page to page
\interfootnotelinepenalty=10000
% Allow todonotes inside footnotes without blowing up LaTeX
% Next command works but now notes can overlap. Instead, we'll define 
% a special footnote note command that performs this redefinition.
%\renewcommand{\marginpar}{\marginnote}%

% Save original definition of \marginpar
\let\oldmarginpar\marginpar

% Workaround for todonotes problem with natbib (To Do list title comes out wrong)
\makeatletter\let\chapter\@undefined\makeatother % Undefine \chapter for todonotes

% Define note commands
\newcommand{\smalltodo}[2][] {\todo[caption={#2}, size=\scriptsize, fancyline, #1] {\begin{spacing}{0.5}#2\end{spacing}}}
\newcommand{\rhs}[2][]{\smalltodo[color=green!30,#1]{{\bf RS:} #2}}
\newcommand{\rhsnolist}[2][]{\smalltodo[nolist,color=green!30,#1]{{\bf RS:} #2}}
\newcommand{\rhsfn}[2][]{%  To be used in footnotes (and in floats)
\renewcommand{\marginpar}{\marginnote}%
\smalltodo[color=green!30,#1]{{\bf RS:} #2}%
\renewcommand{\marginpar}{\oldmarginpar}}
%\newcommand{\textnote}[1]{\ifnotes{{\noindent\color{red}#1}}{}}
\newcommand{\textnote}[1]{\ifnotes{{\colorbox{yellow}{{\color{red}#1}}}}{}}
\newcommand{\antoine}[1]{\textcolor{red}{\textbf{TBD Antoine: #1}}}
\newcommand{\dimi}[1]{\textcolor{blue}{\textbf{TBD Dimi: #1}}}
\newcommand{\alll}[1]{\textcolor{gray}{\textbf{TBD All: #1}}}


% Command to start a new page, starting on odd-numbered page if twoside option 
% is selected above
\newcommand{\clearRHS}{\clearpage\thispagestyle{empty}\cleardoublepage\thispagestyle{plain}}

% Number paragraphs and subparagraphs and include them in TOC
\setcounter{tocdepth}{2}

% JF-specific includes:

\usepackage{indentfirst} % Indent first sentence of a new section.
%\usepackage{endnotes}    % Use endnotes instead of footnotes
\usepackage{jf}          % JF-specific formatting of sections, etc.
\usepackage[labelfont=bf,labelsep=period]{caption}   % Format figure captions
\captionsetup[table]{labelsep=none}

% Define theorem-like commands and a few random function names.
\newtheorem{condition}{CONDITION}
\newtheorem{corollary}{COROLLARY}
\newtheorem{proposition}{PROPOSITION}
\newtheorem{obs}{OBSERVATION}
\newcommand{\argmax}{\mathop{\rm arg\,max}}
\newcommand{\sign}{\mathop{\rm sign}}
\newcommand{\defeq}{\stackrel{\rm def}{=}}

\usepackage{array}
\usepackage{caption}

\captionsetup[table]{
  labelsep=newline,
  justification=centerfirst
}

\usepackage{stackengine}

\newlength\mytemplength
\newcommand\parboxc[3]{%
    \settowidth{\mytemplength}{#3}%
    \parbox[#1][#2]{\mytemplength}{\centering #3}%
}

\usepackage{bbold}
\begin{document}
 
\begin{table}
\centering
\caption{Single table}
\begin{tabular}{lccccc}
\toprule
 & \multicolumn{2}{c}{\parboxc{c}{0.6cm}{first two}}& \multicolumn{3}{c}{\parboxc{c}{0.6cm}{last three}} \\

 \cmidrule(lr){2-3}\cmidrule(lr){4-6}

 & \parboxc{c}{0.6cm}{(1)} & \parboxc{c}{0.6cm}{(2)} & \parboxc{c}{0.6cm}{(3)} & \parboxc{c}{0.6cm}{(4)} & \parboxc{c}{0.6cm}{(5)} \\
\midrule
C & \phantom{*}0.2974*** & \phantom{*}0.3024*** & \phantom{*}0.2746** & \phantom{*}0.2746** & \phantom{*}0.2746** \\
 & (0.0966)\phantom{**} & (0.0899)\phantom{**} & (0.1046)\phantom{*} & (0.1046)\phantom{*} & (0.1046)\phantom{*}\smallskip \\
D &   & \phantom{*}0.0163\phantom{*} &   &   &   \\
 &   & (0.0924) &   &   &  \smallskip \\
A & \phantom{*}0.2789** & \phantom{*}0.2842*** & \phantom{*}0.2878*** & \phantom{*}0.2878*** & \phantom{*}0.2878*** \\
 & (0.1077)\phantom{*} & (0.1081)\phantom{**} & (0.1091)\phantom{**} & (0.1091)\phantom{**} & (0.1091)\phantom{**}\smallskip \\
B & \phantom{*}0.0724\phantom{*} & \phantom{*}0.0578\phantom{*} & \phantom{*}0.0594\phantom{*} & \phantom{*}0.0594\phantom{*} & \phantom{*}0.0594\phantom{*} \\
 & (0.0985) & (0.0986) & (0.0999) & (0.0999) & (0.0999) \\
\medskip\\
area FE & Yes & Yes & Yes & Yes & Yes \\
I FE & Yes & No & No & Yes & Yes \\
\medskip\\
Market controls & 3 & 1 & 5 & 1 & 2 \\
\medskip\\
Observations & 100 & 100 & 100 & 100 & 100 \\
$R^2$ & 0.056800 & 0.040500 & 0.032800 & 0.032800 & 0.032800 \\
\bottomrule
\end{tabular}

\end{table}

\newpage
 
\begin{table}
\centering
\caption{Table with three panels}
\begin{tabular}{lccccc}
\toprule \multicolumn{6}{c}{\parboxc{c}{0.7cm}{Panel C: final short regressions}} \\\hline
 & \multicolumn{2}{c}{\parboxc{c}{0.6cm}{first two}}& \multicolumn{3}{c}{\parboxc{c}{0.6cm}{last three}} \\

 \cmidrule(lr){2-3}\cmidrule(lr){4-6}

 & \parboxc{c}{0.6cm}{(1)} & \parboxc{c}{0.6cm}{(2)} & \parboxc{c}{0.6cm}{(3)} & \parboxc{c}{0.6cm}{(4)} & \parboxc{c}{0.6cm}{(5)} \\
\midrule
C & \phantom{*}0.2974*** & \phantom{*}0.3024*** & \phantom{*}0.2746** & \phantom{*}0.2746** & \phantom{*}0.2746** \\
 & (0.0966)\phantom{**} & (0.0899)\phantom{**} & (0.1046)\phantom{*} & (0.1046)\phantom{*} & (0.1046)\phantom{*}\smallskip \\
D &   & \phantom{*}0.0163\phantom{*} &   &   &   \\
 &   & (0.0924) &   &   &  \smallskip \\
A & \phantom{*}0.2789** & \phantom{*}0.2842*** & \phantom{*}0.2878*** & \phantom{*}0.2878*** & \phantom{*}0.2878*** \\
 & (0.1077)\phantom{*} & (0.1081)\phantom{**} & (0.1091)\phantom{**} & (0.1091)\phantom{**} & (0.1091)\phantom{**}\smallskip \\
B & \phantom{*}0.0724\phantom{*} & \phantom{*}0.0578\phantom{*} & \phantom{*}0.0594\phantom{*} & \phantom{*}0.0594\phantom{*} & \phantom{*}0.0594\phantom{*} \\
 & (0.0985) & (0.0986) & (0.0999) & (0.0999) & (0.0999) \\
\medskip\\
area FE & Yes & Yes & Yes & Yes & Yes \\
I FE & Yes & No & No & Yes & Yes \\
\medskip\\
Market controls & 3 & 1 & 5 & 1 & 2 \\
\medskip\\
Observations & 100 & 100 & 100 & 100 & 100 \\
$R^2$ & 0.056800 & 0.040500 & 0.032800 & 0.032800 & 0.032800 \bigskip \\ 


\hline \multicolumn{6}{c}{\parboxc{c}{0.6cm}{Panel B: short regressions}} \\\hline
 & \parboxc{c}{0.6cm}{(1)} & \parboxc{c}{0.6cm}{(2)} & \parboxc{c}{0.6cm}{(3)} & \parboxc{c}{0.6cm}{(4)} & \parboxc{c}{0.6cm}{(5)} \\
\midrule
C & \phantom{*}0.2974*** & \phantom{*}0.3024*** & \phantom{*}0.2746** & \phantom{*}0.2746** & \phantom{*}0.2746** \\
 & (0.0966)\phantom{**} & (0.0899)\phantom{**} & (0.1046)\phantom{*} & (0.1046)\phantom{*} & (0.1046)\phantom{*} \\
\medskip\\
Observations & 100 & 100 & 100 & 100 & 100 \\
$R^2$ & 0.056800 & 0.040500 & 0.032800 & 0.032800 & 0.032800 \bigskip \\ 


\hline \multicolumn{6}{c}{\parboxc{c}{0.6cm}{Panel C: final short regressions}} \\\hline
 & \multicolumn{1}{c}{\parboxc{c}{0.6cm}{One col}}& \multicolumn{1}{c}{\parboxc{c}{0.6cm}{another}}& \multicolumn{1}{c}{\parboxc{c}{0.6cm}{one more}}& \multicolumn{2}{c}{\parboxc{c}{0.6cm}{the rest}} \\

 \cmidrule(lr){2-2}\cmidrule(lr){3-3}\cmidrule(lr){4-4}\cmidrule(lr){5-6}

 & \parboxc{c}{0.6cm}{(1)} & \parboxc{c}{0.6cm}{(2)} & \parboxc{c}{0.6cm}{(3)} & \parboxc{c}{0.6cm}{(4)} & \parboxc{c}{0.6cm}{(5)} \\
\midrule
D &   & \phantom{*}0.0163\phantom{*} &   &   &   \\
 &   & (0.0924) &   &   &   \\
\medskip\\
area FE & Yes & Yes & Yes & Yes & Yes \\
I FE & Yes & No & No & Yes & Yes \\
\medskip\\
Market controls & 3 & 1 & 5 & 1 & 2 \\
\medskip\\
Observations & 100 & 100 & 100 & 100 & 100 \\
$R^2$ & 0.056800 & 0.040500 & 0.032800 & 0.032800 & 0.032800 \\
\bottomrule
\end{tabular}

\end{table}

\newpage
\begin{table}
\centering
\caption{Renamed table}
\begin{tabular}{lccccc}
\toprule
 & \multicolumn{2}{c}{\parboxc{c}{0.6cm}{first two}}& \multicolumn{3}{c}{\parboxc{c}{0.6cm}{last three}} \\

 \cmidrule(lr){2-3}\cmidrule(lr){4-6}

 & \parboxc{c}{0.6cm}{(1)} & \parboxc{c}{0.6cm}{(2)} & \parboxc{c}{0.6cm}{(3)} & \parboxc{c}{0.6cm}{(4)} & \parboxc{c}{0.6cm}{(5)} \\
\midrule
main variable & \phantom{*}0.2974*** & \phantom{*}0.3024*** & \phantom{*}0.2746** & \phantom{*}0.2746** & \phantom{*}0.2746** \\
 & (0.0966)\phantom{**} & (0.0899)\phantom{**} & (0.1046)\phantom{*} & (0.1046)\phantom{*} & (0.1046)\phantom{*}\smallskip \\
D &   & \phantom{*}0.0163\phantom{*} &   &   &   \\
 &   & (0.0924) &   &   &  \smallskip \\
$\gamma$ & \phantom{*}0.2789** & \phantom{*}0.2842*** & \phantom{*}0.2878*** & \phantom{*}0.2878*** & \phantom{*}0.2878*** \\
 & (0.1077)\phantom{*} & (0.1081)\phantom{**} & (0.1091)\phantom{**} & (0.1091)\phantom{**} & (0.1091)\phantom{**}\smallskip \\
B & \phantom{*}0.0724\phantom{*} & \phantom{*}0.0578\phantom{*} & \phantom{*}0.0594\phantom{*} & \phantom{*}0.0594\phantom{*} & \phantom{*}0.0594\phantom{*} \\
 & (0.0985) & (0.0986) & (0.0999) & (0.0999) & (0.0999) \\
\medskip\\
area FE & Yes & Yes & Yes & Yes & Yes \\
I FE & Yes & No & No & Yes & Yes \\
\medskip\\
Market controls & 3 & 1 & 5 & 1 & 2 \\
\medskip\\
Observations & 100 & 100 & 100 & 100 & 100 \\
$R^2$ & 0.056800 & 0.040500 & 0.032800 & 0.032800 & 0.032800 \\
\bottomrule
\end{tabular}

\end{table}

\newpage
\begin{table}
\centering
\caption{Table with bottom blocks}
\begin{tabular}{lccccc}
\toprule
 & \multicolumn{3}{c}{\parboxc{c}{0.6cm}{21 firms}}& \multicolumn{2}{c}{\parboxc{c}{0.6cm}{Five firm only}} \\

 \cmidrule(lr){2-4}\cmidrule(lr){5-6}

 & \parboxc{c}{0.6cm}{(1)} & \parboxc{c}{0.6cm}{(2)} & \parboxc{c}{0.6cm}{(3)} & \parboxc{c}{0.6cm}{(4)} & \parboxc{c}{0.6cm}{(5)} \\
\midrule
D &   & \phantom{*}0.0163\phantom{*} &   &   &   \\
 &   & (0.0924) &   &   &   \\
\medskip\\
area FE & Yes & Yes & Yes & Yes & Yes \\
I FE & Yes & No & No & Yes & Yes \\
\medskip\\
Market controls & 3 & 1 & 5 & 1 & 2 \\
\medskip\\
Observations & 100 & 100 & 100 & 100 & 100 \\
$R^2$ & 0.056800 & 0.040500 & 0.032800 & 0.032800 & 0.032800 \\
\medskip\\
Nb Firm & 21 & 21 & 21 & 5 & 5 \\
\bottomrule
\end{tabular}

\end{table}

\newpage
\begin{table}
\centering
\caption{Custom parameters}
\begin{tabular}{lccccc}
\hline
 & \multicolumn{2}{c}{\parboxc{c}{0.6cm}{first two}}& \multicolumn{3}{c}{\parboxc{c}{0.6cm}{last three}} \\

 \cmidrule(lr){2-3}\cmidrule(lr){4-6}

 & \parboxc{c}{0.6cm}{(1)} & \parboxc{c}{0.6cm}{(2)} & \parboxc{c}{0.6cm}{(3)} & \parboxc{c}{0.6cm}{(4)} & \parboxc{c}{0.6cm}{(5)} \\
\hline
main variable & \phantom{*}0.297** & \phantom{*}0.302** & \phantom{*}0.275* & \phantom{*}0.275* & \phantom{*}0.275* \\
 & (0.003)\phantom{*} & (0.001)\phantom{*} & (0.01)\phantom{*} & (0.01)\phantom{*} & (0.01)\phantom{*}\smallskip \\
D & (-) & \phantom{*}0.016 & (-) & (-) & (-) \\
 & (-) & (0.86) & (-) & (-) & (-)\smallskip \\
$\gamma$ & \phantom{*}0.279* & \phantom{*}0.284** & \phantom{*}0.288** & \phantom{*}0.288** & \phantom{*}0.288** \\
 & (0.011) & (0.01)\phantom{**} & (0.01)\phantom{**} & (0.01)\phantom{**} & (0.01)\phantom{**}\smallskip \\
B & \phantom{*}0.072\phantom{*} & \phantom{*}0.058 & \phantom{*}0.059\phantom{*} & \phantom{*}0.059\phantom{*} & \phantom{*}0.059\phantom{*} \\
 & (0.464) & (0.56) & (0.553) & (0.553) & (0.553) \\
\smallskip\\
area FE & Yes & Yes & Yes & Yes & Yes \\
I FE & Yes & No & No & Yes & Yes \\
\smallskip\\
Market controls & 3 & 1 & 5 & 1 & 2 \\
\smallskip\\
$R^2$ & 0.060000 & 0.040000 & 0.030000 & 0.030000 & 0.030000 \\
\hline
\end{tabular}

\end{table}



\end{document}
